\documentclass[a4paper]{article}

%% Language and font encodings
\usepackage[french]{babel}
\usepackage[utf8x]{inputenc}
\usepackage[T1]{fontenc}
\usepackage{pdflscape}
\usepackage{xspace}
\usepackage[table,xcdraw]{xcolor}
\usepackage{multirow}
\usepackage{float}
\usepackage{titlesec}
\usepackage{lipsum}

%% Sets page size and margins
\usepackage[a4paper,top=3cm,bottom=2cm,left=3.5cm,right=3.5cm,marginparwidth=1.75cm]{geometry}

%% Useful packages
\usepackage{amsmath}
\usepackage{amssymb}
\usepackage{amsfonts}
\usepackage{amsthm}
\usepackage{graphicx}
\usepackage[colorinlistoftodos]{todonotes}
\usepackage[colorlinks=true, allcolors=blue]{hyperref}
\usepackage[linesnumbered, ruled, vlined]{algorithm2e}
\usepackage{algorithmic}
\usepackage{etoolbox}

\patchcmd{\thebibliography}{\section*}{\section*}{}{}
\patchcmd{\thetoc}{\section*}{\section*}{}{}

\titleformat{\section}
  {\normalfont\LARGE\bfseries}{\thesection}{1em}{}
\titlespacing*{\section}{0pt}{3.5ex plus 1ex minus .2ex}{2.3ex plus .2ex}


\begin{document}

\begin{titlepage}
\begin{center}
 {\Huge \bfseries Système de gestion de données distribué\\}
 \vspace{1cm}
 {\Large \bfseries Rapport de projet \\}
 % ----------------------------------------------------------------
 \vspace{2cm}
 {\Large Cours de systèmes distribués \\ M1 Informatique Fondamentale \\ ENS de Lyon \\ Printemps 2018 \\}

 \vspace{2cm}

{\Large \urlstyle{same} \color{black}
	\href{mailto:guillaume.coiffier@ens-lyon.fr}{guillaume.coiffier@ens-lyon.fr}\\
}

\vfill
L'intégralité du code source est disponible à l'adresse suivante : \\
\url{https://github.com/GCoiffier/Distributed-Data-Manager-Erlang}

\end{center}
\tableofcontents
\vspace{4cm}
\end{titlepage}

\section{Description du programme}
\label{chap:description}

\subsection{Topologie du réseau}

Nous avons fait le choix d'imposer une topologie précise à notre réseau, plutôt que de gérer une topologie arbitraire.
Les conséquences de ce choix sont discutées en chapitre \ref{chap:discussion}. \\
On distinguera, dans le réseau, trois types d'agents :
\begin{itemize}
\item Le noeud \textbf{master}, responsable des connections entrantes au réseau.
\item Les noeuds de requête (\textbf{query}), responsables de la gestion des
requêtes de stockage et d'accès aux données.
\item Les noeuds de stockage (\textbf{storage}) responsables... du stockage des données.
\end{itemize}

Le noeud \textbf{master} est relié à l'intégralité des noeuds \textbf{query}. Ces derniers
forment une clique. Dans la pratique, ils ont simplement accès à la liste des Pid de tous les autres noeuds query. \
Enfin, les noeuds de stockages sont reliés à un ou plusieurs noeuds \textbf{query} et forment leurs fils.

\section{Initialisation du serveur}

Le noeud master (module \texttt{server.erl}) est responsable de l'Initialisation du serveur. Il est lancé lorsque l'on
compile le module \texttt{server}. Il a la tâche d'initialiser N noeuds de requêtes, qui vont à leur tour initialiser M noeuds de stockage chacun. \\
Les valeurs par défaut de N et M sont respectivement 10 et 5.

\section{Méthode de stockage}
Le stockage des données est assuré par un

\chapter{Fonctionnement du programme}
\label{chap:fonctionnement}

\section{Interface client}

L'interface du client est décrite par le module \texttt{client.erl}. Elle comprend les fonctions suivantes :
\begin{itemize}
    \item \texttt{connect/1}
    \item \texttt{send\_data/2}
    \item \texttt{fetch\_data/}
    \item \texttt{release\_data}
\end{itemize}

\section{Gestion des requêtes d'un client par les noeuds de requête}

\chapter{Discussion sur les performances et la robustesse de la solution choisie}
\label{chap:discussion}

\section{Robustesse du réseau}

\section{Complexité de communication}

\end{document}
